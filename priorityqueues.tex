%========== Priority Queues ==========%

\chapter{Priority Queues}
\label{ch:priorityqueues}

\textbf{Relevant Assignment} Problem ?-?\\\\
\textbf{Keywords} Min- and maxheap, stacks
\vspace{1in}

\noindent A priority queue is an abstract data structure, which sorts and
maintains a set of data in such a way, that the data is prioritized. It is an
important distinction to make that a priority queue isn't a particular data
structure, but rather a class of data structures. Data structures that work
well with this notion include stacks, heaps, self-organizing lists, etc.
We will focus mainly on heaps.

\section{Heap}
...

\section{Stack}
% extra stuff, don't focus on this, I think.
A stack is one of the simplest queuing data structures. [...]

