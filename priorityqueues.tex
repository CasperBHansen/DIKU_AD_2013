%========== Priority Queues ==========%

\chapter{Priority Queues}
\label{ch:priorityqueues}

\textbf{Relevant Assignment} Problem ?-?\\\\
\textbf{Pensum}CLRS Ch. 6\\\\
\textbf{Keywords} Min- and maxheap, stacks
\vspace{1in}

\noindent A priority queue is an abstract data structure, which sorts and
maintains a set of data in such a way, that the data is prioritized. It is an
important distinction to make that a priority queue isn't a particular data
structure, but rather a class of data structures. Data structures that work
well with this notion include stacks, heaps, self-organizing lists, etc.
We will focus mainly on heaps.

\section{Heap}
A heap is a nearly complete $n$-nary tree, we will concern ourselves only with
binary heaps - and as such, a binary heap is a nearly complete binary tree. A
heap, in the sense of priority queues implements a set $S$ of elements, in
which each element has an associated key. The actual structure of a heap is
simply an array, and so a heap is defined by the operations performed on an
array, rather than the data structure itself. The reason a heap is a
tree-structure is a visualization of the data, but not reflected by the data
structure in any way.

% TODO: properties of a heap, such as how to get a parent node, each of its
% children, etc.

A heap, however, is not a priority queue in and of itself, it must adhere to
a set of properties, and that is either the max- or min-heap properties.

\begin{description}
	\item \textbf{Max-Heap} ...
	\item \textbf{Min-Heap} ...
\end{description}
...

\section{Stack}
% extra stuff, don't focus on this, I think.
A stack is one of the simplest queuing data structures. [...]

