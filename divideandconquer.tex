%========== Divide & Conquer ==========%

\chapter{Divide \& Conquer}
\label{ch:divideandconquer}

\textbf{Relevant Assignment} Problem ?-?\\\\
\textbf{Algorithms} Merge sort, quicksort\\\\
\textbf{Keywords} The master method, recursion trees (validation by induction)
\vspace{1in}

\noindent Divide and conquer is paradigm of algorithmic methodologies.
% more?
\\\\
\noindent \textbf{Divide} the problem into a number of subproblems that are
smaller instances of the same problem.
\\\\
\noindent \textbf{Conquer} the subproblems by solving them recursively. If the
subproblem sizes are small enough, however, just solve the subproblems in a
straightforward manner.
\\\\
\noindent \textbf{Combine} the solutions to the subproblems into the solution
for the original problem.
\\\\

\section{Recursion Trees}
A recursion tree is a pictorial representation of an algorithms recursive
calls. That is, we simply draw the recursions as they occur in a
tree-structure, such that we may [...]

Although recursion trees are useful for getting an idea about the complexity
of a recursive algorithm, it doesn't directly give us any concrete about this.
For that must employ more strict methods that are rooted in mathematics, and
not drawings.

\section{Solving Recurrences}
...

\section{Substitution}
...

\section{The Master Method}
...

