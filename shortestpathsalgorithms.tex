%========== Shortest Paths Algorithms ==========%

\chapter{Shortest Paths Algorithms}
\label{ch:shortestpathsalgorithms}

\textbf{Pensum} 24 \cite{clrs} \\\\
\textbf{Assignments} 13-3 \\\\
\textbf{Algorithms} Single-source, Dijkstra, Bellman-Ford, breadth-first
search \\\\
\textbf{Keywords} Triangle inequality, relaxation, optimal substructure,
represenation invariant
\vspace{1in}

\noindent We are given a weighted, directed graph $G = (V, E)$, with weight
function $w : E \rightarrow \mathbb{R}$, mapping the edges of the graph to
real-valued weights. The \textit{weight} $w(p)$ of a \textit{path} $p =
\langle v_0, v_1, \dots, v_k \rangle$ is the sum of the weights of its
constituent edges.
\begin{align}
	w(p) = \sum_{k}^{i=1} w(v_{i-1}, v_i)
\end{align}
We define the shortest path weight $\delta(u, v)$ from $u$ to $v$ by
\begin{align}
	\delta(u, v) =
	\begin{cases}
		min\{w(p): u \rightarrow v\}
		& \textnormal{if there is a path from $u$ to $v$,} \\
		\infty & \textnormal{otherwise.}
	\end{cases}
\end{align}
A \textit{shortest path} $p$ from $v$ to $u$ is defined as $w(p) =
\delta(u, v)$.

\section{Properties}
% NOTE: maybe proofs of lemmas is going a little over the top? Some are very long.
We give the properties of shortest paths.

\subsection{The Triangle Inequality}
For any edge $(u, v) \in E$, we have $\delta(s, v) \leq \delta(s, u) +
w(u, v)$ - meaning that a shortest path fulfills the triangle inequality (see
appendix \ref{appendix:equations|eqn:triangle-inequality}).
\begin{lemma}
	\textbf{Triangle inequality property} \\
	...
\end{lemma}

\begin{proof} \textnormal{\cite[p.~671, thm. 24.10]{clrs}} \\
	...
\end{proof}

\subsection{Upper-bound}
We always have $v.d \geq \delta(s, v)$ for all vertices $v \in V$, and once
$v.d = \delta(s, v)$, it never changes.
\begin{lemma}
	\textbf{Upper-bound property} \\
	...
\end{lemma}

\begin{proof} \textnormal{\cite[p.~671-672, thm. 24.11]{clrs}} \\
	...
\end{proof}

\subsection{No-path}
If there is no path from $s$ to $v$, then we always have $v.d = \delta(s, v) =
\infty$.
\begin{corollary}
	\textbf{No-path property} \\
	...
\end{corollary}

\begin{proof} \textnormal{\cite[p.~672, thm. 24.12]{clrs}} \\
	...
\end{proof}

\subsection{Convergence}
...
\begin{lemma}
	\textbf{Convergence property} \\
	...
\end{lemma}

\begin{proof} \textnormal{\cite[p.~672-673, thm. 24.14]{clrs}} \\
	...
\end{proof}

\subsection{Path-relaxation}
...
\begin{lemma}
	\textbf{Path-relaxation property} \\
	...
\end{lemma}

\begin{proof} \textnormal{\cite[p.~673, thm. 24.15]{clrs}} \\
	...
\end{proof}

\subsection{Predecessor-subgraph}
Once $v.d = \delta(s, v)$ for all $v \in V$, the predecessor subgraph is a
shortest-paths tree rooted at $s$.
\begin{lemma}
	\textbf{Predecessor-subgraph property} \\
	...
\end{lemma}

\begin{proof} \textnormal{\cite[p.~676, thm. 24.17]{clrs}} \\
	...
\end{proof}

\newpage
\section{Optimal Substructure}
Shortest paths algorithms typically rely on the property that a shortest path
between two vertices contains other shortest paths within it. This is one of
the key indicators of \textit{dynamic programming} (see chapter
\ref{ch:dynamicprogramming}) and \textit{greedy} (see chapter
\ref{ch:greedyalgorithms}) problems.

\begin{lemma}
	\textbf{Subpaths of shortest paths are shortest paths} \\
	...
\end{lemma} 

\begin{proof} \textnormal{\cite[p.~645, thm.~24.1]{clrs}} \\
	...
\end{proof}

\section{Single-source Shortest Path}
% ch. 24, CLRS
Given a graph $G = (V, E)$, we want to find a shortest path from a given
\textit{source} vertex $s \in V$ to each vertex $v \in V$.
\\\\
\noindent We can reduce many other problems to the problem defined above.
\begin{description}
	\item \textbf{Single-destination} Find a shortest path to a given
\textit{destination} vertex $t$ from a vertex $v$ - this is the opposite
problem. By reversing the edge directions we effectively reduced it to a
source-source problem.
	\item \textbf{Single-pair} Find a shortest path from $u$ to $v$. If we let
$u$ be the source vertex, then this is the exact same problem. All known
algorithms for this problem have the same worst-case asymptotic running-time
as the best single-source algorithms. % TODO: Elaborate on why, prove it maybe!
	\item \textbf{All-pair} Find a shortest path from $u$ to $v$ for every
pair of vertices $u$ and $v$. This is essentially the single-source problem,
but for all vertices in the set as the source.
\end{description}

\newpage
\section{Bellman-Ford's Algorithm}
% p. 651, CLRS
...

\subsection{Analysis}
...

\newpage
\section{Dijkstra's Algorithm}
% p. 658, CLRS
...
\begin{algorithm}
	\caption{Dijkstra's algorithm}
	\label{alg:dijkstra}
	
	\SetKwInOut{Input}{Input}
	\SetKwInOut{Output}{Output}
	
	\SetKwFunction{Dijkstra}{Dijkstra}
	\SetKwFunction{InitSingleSource}{InitializeSingleSource}
	\SetKwFunction{ExtractMin}{ExtractMin}
	\SetKwFunction{Relax}{Relax}
	
	\Input{A graph $G$, a weight function $w$ and a source vertex $s$.}
	\Output{...}
	
	\BlankLine
	\Dijkstra($G$, $w$, $s$) \\
	\Begin
	{
		\InitSingleSource($G$, $s$) \\
		$S = \emptyset$ \\
		$Q = G.V$ \\
		\While{$Q \neq \emptyset$}
		{
			$u = $ \ExtractMin($Q$) \\
			$S = S \cup \{u\}$ \\
			\For{\textnormal{each vertex } $v \in G.Adj[u]$}
			{
				\Relax($u$, $v$, $w$)
			}
		}
	}
\end{algorithm}

\subsection{Analysis}
...

