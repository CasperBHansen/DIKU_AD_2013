%========== Shortest Paths Algorithms ==========%

\chapter{Shortest Paths Algorithms}
\label{ch:shortestpathsalgorithms}

\textbf{Pensum} 24 \cite{clrs} \\\\
\textbf{Assignments} 13-3 \\\\
\textbf{Algorithms} Single-source, Dijkstra, Bellman-Ford, breadth-first
search \\\\
\textbf{Keywords} Triangle inequality, relaxation, optimal substructure,
represenation invariant
\vspace{1in}

\noindent We are given a weighted, directed graph $G = (V, E)$, with weight
function $w : E \rightarrow \mathbb{R}$, mapping the edges of the graph to
real-valued weights. The \textit{weight} $w(p)$ of a \textit{path} $p =
\langle v_0, v_1, \dots, v_k \rangle$ is the sum of the weights of its
constituent edges.
\begin{align}
	w(p) = \sum_{k}^{i=1} w(v_{i-1}, v_i)
\end{align}
We define the shortest path weight $\delta(u, v)$ from $u$ to $v$ by
\begin{align}
	\delta(u, v) =
	\begin{cases}
		min\{w(p): u \rightarrow v\}
		& \textnormal{if there is a path from $u$ to $v$,} \\
		\infty & \textnormal{otherwise.}
	\end{cases}
\end{align}
A \textit{shortest path} $p$ from $v$ to $u$ is defined as $w(p) =
\delta(u, v)$.

\section{Properties}
We give the properties of shortest paths.

\begin{description}
	\item \textbf{The Triangle Inequality} \cite[p.~671, thm. 24.10]{clrs} \\
For any edge $(u, v) \in E$, we have $\delta(s, v) \leq \delta(s, u) +
w(u, v)$ - meaning that a shortest path fulfills the triangle inequality (see
appendix \ref{appendix:equations|eqn:triangle-inequality}).

	\item \textbf{Upper-bound} \cite[p.~671-672, thm. 24.11]{clrs} \\
We always have $v.d \geq \delta(s, v)$ for all vertices $v \in V$, and once
$v.d = \delta(s, v)$, it never changes.

	\item \textbf{No-path} \cite[p.~672, thm. 24.12]{clrs} \\
If there is no path from $s$ to $v$, then we always have $v.d = \delta(s, v) =
\infty$.

	\item \textbf{Convergence} \cite[p.~672-673, thm. 24.14]{clrs} \\
...
	\item \textbf{Path-relaxation} \cite[p.~673, thm. 24.15]{clrs} \\
If $p = \langle v_0, v_1, \dots, v_k \rangle$ is a shortest path from
$s = v_0$ to $v_k$, and we relax the edge of $p$ in order $(v_0, v_1),
(v_1, v_2), \dots, (v_{k-1}, v_k)$, then $v_k.d = \delta(s, v_k)$. This
property holds regardless of any other relaxation steps that occur, even if
they are intermixed with relaxations of the edges of $p$.
	\item \textbf{Predecessor-subgraph} \cite[p.~676, thm. 24.17]{clrs} \\
Once $v.d = \delta(s, v)$ for all $v \in V$, the predecessor subgraph is a
shortest-paths tree rooted at $s$.
\end{description}

\section{Optimal Substructure}
Shortest paths algorithms typically rely on the property that a shortest path
between two vertices contains other shortest paths within it. This is one of
the key indicators of \textit{dynamic programming} (see chapter
\ref{ch:dynamicprogramming}) and \textit{greedy} (see chapter
\ref{ch:greedyalgorithms}) problems.

\begin{lemma}
	\textbf{Subpaths of shortest paths are shortest paths} \\
	...
\end{lemma} 

\begin{proof} \textnormal{\cite[p.~645, thm.~24.1]{clrs}} \\
	...
\end{proof}

\newpage
\section{Single-source Shortest Path}
% ch. 24, CLRS
Given a graph $G = (V, E)$, we want to find a shortest path from a given
\textit{source} vertex $s \in V$ to each vertex $v \in V$.
\\\\
\noindent We can reduce many other problems to the problem defined above.
\begin{description}
	\item \textbf{Single-destination} Find a shortest path to a given
\textit{destination} vertex $t$ from a vertex $v$ - this is the opposite
problem. By reversing the edge directions we effectively reduced it to a
source-source problem.
	\item \textbf{Single-pair} Find a shortest path from $u$ to $v$. If we let
$u$ be the source vertex, then this is the exact same problem. All known
algorithms for this problem have the same worst-case asymptotic running-time
as the best single-source algorithms. % TODO: Elaborate on why, prove it maybe!
	\item \textbf{All-pair} Find a shortest path from $u$ to $v$ for every
pair of vertices $u$ and $v$. This is essentially the single-source problem,
but for all vertices in the set as the source.
\end{description}

\subsection{Subprocedures}
We will make use of a number of subprocedures in the following algorithms.
\begin{minipage}[t]{0.45\linewidth}
	\begin{algorithm}[H]
		\caption{Init Single-source}
		\label{alg:init-single-source}
		
		\SetKwInOut{In}{Input}
		\SetKwInOut{Out}{Output}
		\SetKwInOut{Time}{Complexity}
		\SetKw{Nil}{NIL}
		
		\SetKwFunction{InitSingleSource}{InitSingleSource}
		
%		\In{A graph $G$ and a source vertex $s$.}
%		\Out{Initializes the nodes attributes of the graph $G$.}
		\Time{$\Theta(V)$-time.}
		
		\BlankLine
		\InitSingleSource($G$, $s$) \\
		\Begin
		{
			\For{\textnormal{each vertex } $v \in G.V$}
			{
				$v.d = \infty$ \\
				$v.\pi = $ \Nil
			}
			$s.d = 0$
		}
	\end{algorithm}
\end{minipage}
\hspace{0.5cm}
\begin{minipage}[t]{0.45\linewidth}
	\begin{algorithm}[H]
		\caption{Relax edge}
		\label{alg:relax}
		
		\SetKwInOut{In}{Input}
		\SetKwInOut{Out}{Output}
		\SetKwInOut{Time}{Complexity}
		
		\SetKw{Nil}{NIL}
		
		\SetKwFunction{Relax}{Relax}
		
%		\In{Vertices $u$ and $v$, and weight function $w$.}
%		\Out{Relaxes the path to $v$, if going through $u$ is shorter.}
		\Time{$\Theta(1)$-time.}
		
		\BlankLine
		\Relax($u$, $v$, $w$) \\
		\Begin
		{
			\If{$u.d + w(u, v) < v.d$}
			{
				$v.d = u.d + w(u, v)$ \\
				$v.\pi = u$
			}
		}
	\end{algorithm}
\end{minipage}

\newpage
\section{Bellman-Ford Algorithm}
% p. 651, CLRS
The algorithm produces a shortest path if no negative cycles are reachable
from the source, indicating success by returning the boolean value
\texttt{true}, otherwise it returns \texttt{false}. It works by continually
relaxing edges, progressively decreasing the distance estimate on the weight
of a shortest path from the source $s$ to each vertex $v \in V$.
\\\\
\begin{algorithm}[H]
	\caption{Bellman-Ford algorithm}
	\label{alg:bellman-ford}
	
	\SetKwInOut{Input}{Input}
	\SetKwInOut{Output}{Output}
	
	\SetKw{True}{true}
	\SetKw{False}{false}
	
	\SetKwFunction{BellmanFord}{BellmanFord}
	\SetKwFunction{InitSingleSource}{InitializeSingleSource}
	\SetKwFunction{Relax}{Relax}
	
	\Input{A graph $G$, a weight function $w$ and a source vertex $s$.}
	\Output{In-place shortest}
	
	\BlankLine
	\BellmanFord($G$, $w$, $s$) \\
	\Begin
	{
		\InitSingleSource($G$, $s$) \\
		\For{$i = 1$ \KwTo $|G.V| - 1$}
		{
			\For{\textnormal{each edge } $(u, v) \in G.E$}
			{
				\Relax($u$, $v$, $w$)
			}
		}
		\For{\textnormal{each edge } $(u, v) \in G.E$}
		{
			\If{$v.d > u.d + w(u, v)$}{ \Return \False }
		}
		\Return \True
	}
\end{algorithm}

\subsection{Analysis}
The initialization on line 3 takes $O(V)$-time. The loop on lines 4--8 is run
exactly $|V|-1$ times, each pass relaxes every edge once $\Theta(E)$, giving
us the most prominent contribution to the running-time $O(VE)$. The last loop
contributes just $O(E)$.

\newpage
\section{Dijkstra's Algorithm}
% p. 658, CLRS
The algorithm ...
\\\\
\begin{algorithm}[H]
	\caption{Dijkstra's algorithm}
	\label{alg:dijkstra}
	
	\SetKwInOut{Input}{Input}
	\SetKwInOut{Output}{Output}
	
	\SetKwFunction{Dijkstra}{Dijkstra}
	\SetKwFunction{InitSingleSource}{InitializeSingleSource}
	\SetKwFunction{ExtractMin}{ExtractMin}
	\SetKwFunction{Relax}{Relax}
	
	\Input{A graph $G$, a weight function $w$ and a source vertex $s$.}
	\Output{...}
	
	\BlankLine
	\Dijkstra($G$, $w$, $s$) \\
	\Begin
	{
		\InitSingleSource($G$, $s$) \\
		$S = \emptyset$ \\
		$Q = G.V$ \\
		\While{$Q \neq \emptyset$}
		{
			$u = $ \ExtractMin($Q$) \\
			$S = S \cup \{u\}$ \\
			\For{\textnormal{each vertex } $v \in G.Adj[u]$}
			{
				\Relax($u$, $v$, $w$)
			}
		}
	}
\end{algorithm}

\subsection{Analysis}
...

