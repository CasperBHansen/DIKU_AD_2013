%========== Greedy Algorithms ==========%

\chapter{Greedy Algorithms}
\label{ch:greedyalgorithms}

\textbf{Pensum} 16 \cite{clrs} \\\\
\textbf{Assignments} 16-1 \\\\
\textbf{Algorithms} Activity selection, huffman trees \\\\
\textbf{Keywords} The greedy choice, local- and global optimal solution
\vspace{1in}

\noindent Greedy algorithms follow an algorithmic methodology that makes a
series of choices under the assumption that the choice that seems best at the
moment (a \textit{locally optimal solution}) will produce \textit{globally
optimal solution} - this isn't always the case, but in some cases we can prove
that it does.
\\\\
\noindent \textbf{Dynamic Programming} \\
Although greedy problems often exhibit the same characteristics as a dynamic
programming problem, the key difference between these two algorithmic
techniques is that dynamic programming problems tries all possible solutions,
and hence always yields an optimal solution. This is not necessarily the case
for a greedy algorithm, as it does not concern itself with the subproblem, but
rather makes a locally optimal choice \textit{in the hope} that this will lead
to a globally optimal solution - we can, however, show that it will for some
problems.

\newpage
\section{The Greedy Strategy}
Here we list the procedure one must follow in order to design a greedy
algorithm. Each point will be discussed in depth under its own section.
\begin{enumerate}
	\item \textbf{Describe} the optimization problem as one in which making a
choice, we are left with one subproblem.
	\item \textbf{Prove} that the greedy choice always yields a globally
optimal solution.
	\item \textbf{Demonstrate} optimal substructure by showing that, having
made a greedy choice, what remains is a subproblem with the property that if
we combine an optimal solution with the greedy greedy choice, we arrive at an
optimal solution to the original problem.
\end{enumerate}

\subsection{Describing the Optimization Problem}
% p. ?-?, CLRS
...

\subsection{Proving The Greedy Choice}
% p. 424, CLRS
As the greedy choice is not always the optimal choice, you will have to prove the choice for each algorithme.
The proof examines
a globally optimal solution to some subproblem. It then shows how to modify
the solution to substitute the greedy choice for some other choice, resulting in one
similar, but smaller, subproblem.

\subsection{Demonstrating Optimal Substructure}
% p. ?-?, CLRS
...

\section{Huffman}
% p. 428-435, CLRS
...

