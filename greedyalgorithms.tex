%========== Greedy Algorithms ==========%

\chapter{Greedy Algorithms}
\label{ch:greedyalgorithms}

\textbf{Relevant Assignment} Problem 16-1 \\\\
\textbf{Pensum}CLRS Ch. 16\\\\
\textbf{Algorithms} Rod-cutting \\\\
\textbf{Keywords} Optimal solution
\vspace{1in}

\noindent Greedy algorithms follow an algorithmic methodology which [...] 
\\\\
\noindent \textbf{Dynamic Programming} \\
Although greedy problems often exhibit the same characteristics as a dynamic
programming problem, the key difference between these two algorithmic
techniques is that dynamic programming problems tries all possible solutions,
and hence always yields an optimal solution. This is not necessarily the case
for a greedy algorithm, as it does not concern itself with the subproblem, but
rather makes a locally optimal choice \textit{in the hope} that this will lead
to a globally optimal solution - we can, however, show that it will for some
problems.

