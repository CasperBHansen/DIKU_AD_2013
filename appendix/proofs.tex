%========== Proving Algorithms ==========%

\thispagestyle{fancyplain}

\chapter{Proving Algorithms}
\label{ch:proofs}

In this appendix section we give common techniques shared by many, if not all,
algorithmic paradigms.

\section{Loop Invariant}
\label{ch:proofs|sec:loop-invariant}
A loop invariant is based on the notion of mathematical induction, as it
provides a proof of the algorithms correctness in all three of its executed
stages; initialization, maintenance and termination.

The loop invariant is used to prove the correctness of a loop-based algorithm,
and as such we must make sure that when we state a loop invariant the
following is true.
\\\\
\textbf{Initialization}
\label{ch:proofs|sec:loop-invariant|sub:initialization}\\
The loop invariant is true prior to the first iteration of the loop.
\\\\
\textbf{Maintenance}
\label{ch:proofs|sec:loop-invariant|sub:maintenance}\\
If it is true before an iteration of the loop, it remains true before the next
iteration.
\\\\
\textbf{Termination}
\label{ch:proofs|sec:loop-invariant|sub:termination}\\
When the loop terminates, the invariant provides a useful property that helps
to show that the algorithm is correct.

