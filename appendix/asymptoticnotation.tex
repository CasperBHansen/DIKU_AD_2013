%========== Asymptotic Notation ==========%

\thispagestyle{fancyplain}

\chapter{Asymptotic Notation}
\label{ch:asymptoticnotation}
The analysis of the growth of a function is called asymptotic analysis. The
mathematical notation used to describe this analysis is called asymptotic
notation.

\section{$O$}
\label{ch:asymptoticnotation|sec:big-o}
Defines a tight upperbound of a function. That is, $O(n)$ is analogous to the
comparative operator $\leq$. If $O(f(n)) = g(n)$ we say that $f(n)$ is tightly
bounded above by $g(n)$.
\begin{align}
	O(f(n)) =
	\{g(n) \text{ iff } \exists c, n_0 \in \mathbb{R}^{+} :
	0 \geq g(n) \geq c f(n), \forall n \geq n_0 \}
\end{align}

\section{$o$}
\label{ch:asymptoticnotation|sec:little-o}
Defines an upper bound of a function. That is, $o(n)$ is analogous to the
comparative operator $<$. If $o(f(n)) = g(n)$ we say that $f(n)$ is bounded
above by $g(n)$ - this bound is \textit{not} tight.
\begin{align}
	o(f(n)) =
	\{g(n) \text{ iff } \forall c > 0 \exists n_0 > 0 :
	0 \leq c f(n) < c g(n), \forall n \geq n_0 \}
\end{align}

\section{$\Omega$}
\label{ch:asymptoticnotation|sec:big-omega}
Defines a tight lowerbound of a function. That is, $\Omega$ is analogous to
the comparative operator $\geq$. If $\Omega(f(n)) = g(n)$ we say that $f(n)$
is tightly bounded below by $g(n)$.
\begin{align}
	\Omega(f(n)) =
	&\{g(n) \text{ iff } \exists c, n_0 \in \mathbb{R}^{+} :
	0 \geq c f(n) \geq g(n), \forall n \geq n_0 \}
\end{align}

\section{$\omega$}
\label{ch:asymptoticnotation|sec:litte-omega}
Defines a lower bound of a function. That is, $o(n)$ is analogous to the
comparative operator $>$. If $o(f(n)) = g(n)$ we say that $f(n)$ is bounded
below by $g(n)$ - this bound is \textit{not} tight.
\begin{align}
	\omega(f(n)) =
	\{g(n) \text{ iff } \forall c > 0 \exists n_0 > 0 :
	0 \leq c g(n) < f(n), \forall n \geq n_0 \}
\end{align}

\section{$\Theta$}
\label{ch:asymptoticnotation|sec:theta}
Defines a tight bound of a function. That is, it is bounded above and below by
a certain growth. That is, $\Theta(n)$ is analogous to the comparative
operator $=$.
\begin{align}
	\Theta(f(n)) =
	&\{g(n) \text{ iff } \exists c_1, c_2, n_0 \in \mathbb{R}^{+} :\\
	&0 \leq c_1 f(n) \leq g(n) \leq c_2 f(n), \forall n \geq n_0 \}
	\nonumber
\end{align}

\section{Summary}
\label{ch:asymptoticnotation|sec:summary}
In summary, we gather all of the above cases into one equation.
\begin{align}
	f(n) =
	\begin{cases}
		\exists c, n_0 \in \mathbb{R}^{+}
		: 0 \geq g(n) \geq c f(n), \forall n \geq n_0 & O(g(n)) \\
		\exists c, n_0 \in \mathbb{R}^{+}
		: 0 \geq c f(n) \geq g(n), \forall n \geq n_0 & \Omega(g(n)) \\
		\Omega(f(n)) = O(f(n)) & \Theta(f(n))
	\end{cases}
\end{align}

% make a comparison table

