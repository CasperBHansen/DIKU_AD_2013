%========== Assignments ==========%

\thispagestyle{fancyplain}

\chapter{Assignments}
\label{appendix:assignments}
In this appendix section we give an overview of the assignments that can be
asked about during the examination.

\section{2-2 Bubblesort}
\label{appendix:assignments|ass:bubblesort}
The takeaway point of this assignment was mainly that of efficiency, and can
thus be related to how the \textit{divide-and-conquer} paradigm optimizes for
better efficiency. It also served as an exercise for proving \textit{loop
invariants} (see appendix \ref{appendix:proofs|sec:loop-invariant}).

\section{7-1 Hoare Partition}
\label{appendix:assignments|ass:hoare-partition}
This assignment focused on:
\begin{enumerate}
\item Showing how \textit{hoare-partition} runs on a dataset.
\item Proving various parts of the algorithm.
\item Modifying the \textit{quicksort} algorithm to use \textit{hoare-partition} 
instead of the regular \textit{partition}.
\end{enumerate}

\section{15-2 Longest Palindrome Subsequence}
\label{appendix:assignments|ass:longest-palindromic-subsequence}
In this assignment the main problem was to construct an algorithm to find the 
longest palindrome subsequence of a single string, and then prove the running 
time of the algorithm.\\
One solution to this was to use the \textit{LCS} algorithm on the input string 
and its reverse, then splitting that in two and mirroring it, taking into account 
an uneven amount of characters.\\
The runtime of this is heavily based on the runtime of \textit{LCS-LENGTH} and \textit{PRINT-LCS}, as our parts of the solution outside those two algorithms run in linear time.

\section{16-1 Coin Change}
\label{appendix:assignments|ass:coin-change}
...

\section{13-3 AVL-tree}
\label{appendix:assignments|ass:avl-tree}
...