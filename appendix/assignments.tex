%========== Assignments ==========%

\thispagestyle{fancyplain}

\chapter{Assignments}
\label{appendix:assignments}
In this appendix section we give an overview of the assignments that can be
asked about during the examination.

\section{2-2 Bubblesort}
\label{appendix:assignments|ass:bubblesort}
The takeaway point of this assignment was mainly that of efficiency, and can
thus be related to how the \textit{divide-and-conquer} paradigm optimizes for
better efficiency. It also served as an exercise for proving \textit{loop
invariants} (see appendix \ref{appendix:proofs|sec:loop-invariant}).

\section{7-1 Hoare Partition}
\label{appendix:assignments|ass:hoare-partition}
This assignment focused on:
\begin{enumerate}
\item Showing how \textit{hoare-partition} runs on a dataset.
\item Proving that the algorithm doesn't access any array indexes outside the 
input array range.
\item Proving that the returned value \textit{j} is in the range $p\leq j < r$
\item Proving that the algoirithm actually partitions, by having all numbers in 
the range $A[p..j]$ be less than or equal to the numbers in the range $A[j+1..r]$ 
at termination.
\item Modifying the \textit{quicksort} algorithm to use \textit{hoare-partition} 
instead of the regular \textit{partition}, by including the pivot element in one 
of the recursive calls that quicksort makes
\end{enumerate}

\section{15-2 Longest Palindromic Subsequence}
\label{appendix:assignments|ass:longest-palindromic-subsequence}
...

\section{16-1 Coin Change}
\label{appendix:assignments|ass:coin-change}
...

\section{13-3 AVL-tree}
\label{appendix:assignments|ass:avl-tree}
...