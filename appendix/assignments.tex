%========== Assignments ==========%

\thispagestyle{fancyplain}

\chapter{Assignments}
\label{appendix:assignments}
In this appendix section we give an overview of the assignments that can be
asked about during the examination.

\section{2-2 Bubblesort}
\label{appendix:assignments|ass:bubblesort}
The takeaway point of this assignment was mainly that of efficiency, and can
thus be related to how the \textit{divide-and-conquer} paradigm optimizes for
better efficiency. It also served as an exercise for proving \textit{loop
invariants} (see appendix \ref{appendix:proofs|sec:loop-invariant}).

\section{7-1 Hoare Partition}
\label{appendix:assignments|ass:hoare-partition}
This assignment focused on:
\begin{enumerate}
	\item Showing how \textit{hoare-partition} runs on a dataset.
	\item Proving various parts of the algorithm.
	\item Modifying the \textit{quicksort} algorithm to use
\textit{hoare-partition}  instead of the regular \textit{partition}.
\end{enumerate}
The takeaway points were, like the first assignment, mainly to exercise proofs
of algorithms and ensure an understanding of quick sort works.

\newpage
\section{15-2 Longest Palindromic Subsequence}
\label{appendix:assignments|ass:longest-palindromic-subsequence}
Using dynamic programming (see chapter \ref{ch:dynamicprogramming}) we were to
construct the longest palindromic subsequence (LPS). This assignment emphasized
the validity of our proofs, as many fell into the trap of think an LCS
procedure on the original string and the reverse of it would always produce a
LPS --- which is the right train of thought, but can produce non-LPS results.

\section{16-1 Coin Change}
\label{appendix:assignments|ass:coin-change}
The takeaway points here were to reenforce a good understanding of greedy
algorithms (see chapter \ref{ch:greedyalgorithms}), and to exercise proving
optimal substructure as well as overlapping subproblems.

\section{13-3 AVL-tree}
\label{appendix:assignments|ass:avl-tree}
The emphasis of this assignment was on our understanding of binary search
trees in general, and how rebalancing can give us a great deal of performance
in return. Although AVL trees aren't in the syllabus, they can be related to
red-black binary search trees.

