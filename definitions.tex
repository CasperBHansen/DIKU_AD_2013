%========== Definitions ==========%

\thispagestyle{fancyplain}

\chapter{Definitions}
\label{ch:definitions}
In this chapter we will discuss the most important definitions of algorithms
and datastructures.

\section{Asymptotic Notation}
\label{ch:definitions|sec:asymptotic-notation}
The analysis of the growth of a function is called asymptotic analysis. The
mathematical notation used to describe this analysis is called asymptotic
notation.

\subsection{$O$}
\label{ch:definitions|sec:asymptotic-notation|sub:big-o}
Defines a tight upperbounded growth of a function. That is, $O(n)$ is
analogous to the comparison $\leq$.

\subsection{$o$}
\label{ch:definitions|sec:asymptotic-notation|sub:little-o}
...

\subsection{$\Omega$}
\label{ch:definitions|sec:asymptotic-notation|sub:big-omega}
...

\subsection{$\omega$}
\label{ch:definitions|sec:asymptotic-notation|sub:litte-omega}
...

\subsection{$\Theta$}
\label{ch:definitions|sec:asymptotic-notation|sub:theta}
...

\newpage
\section{Loop Invariant}
\label{ch:definitions|sec:loop-invariant}
A loop invariant is based on the notion of mathematical induction, as it
provides a proof of the algorithms correctness in all three of its executed
stages; initialization, maintenance and termination.

The loop invariant is used to prove the correctness of a loop-based algorithm,
and as such we must make sure that when we state a loop invariant the
following is true.

\subsection{Initialization}
The loop invariant is true prior to the first iteration of the loop.

\subsection{Maintenance}
If it is true before an iteration of the loop, it remains true before the next
iteration.

\subsection{Termination}
When the loop terminates, the invariant provides a useful property that helps
to show that the algorithm is correct.

\newpage
\section{Recurrence Solution Methods}
...

\subsection{Recurrence Trees}
...

\subsection{Substitution}
...

\subsection{The Master Method}
...

% \section{Summary}
% \label{ch:definitions|sec:asymptotic-notation|sec:summary}
% TODO: list the definitions in a table

