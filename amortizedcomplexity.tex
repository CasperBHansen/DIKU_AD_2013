%========== Amortized Complexity ==========%

\chapter{Amortized Complexity}
\label{ch:amortizedcomplexity}

\textbf{Relevant Assignment} Week ?, Problem ?-?\\\\
\textbf{Keywords} Aggregate analysis, accounting- and potential method, average.
\vspace{1in}

\noindent In amortized complexity analysis, we average the time required to
perform a sequence of data-structure operations over all the operations
performed. With amortized analysis, we can show that the average cost of an
operation is small, if we average over a sequence of operations, even though
a single operation within the sequence might be expensive.
\\\\
\noindent \textbf{Amortized- and Average-case Analysis} \\
Amortized analysis differs from average-case analysis in that probability is
not involved; an amortized analysis guarantees the \textit{average performance
of each operation in the worst case}.

\newpage
\section{Aggregate Analysis}
In aggregate analysis, we show that for all $n$, a sequence of $n$ operations
takes \textit{worst-case} time $T(n)$ in total. The average cost, or \textbf{
amortized cost}, however, per operation is $\frac{T(n)}{n}$. The amortized
cost applies to each operation, even when there are several types of
operations in the sequence.

\subsection{Example}
...

\section{Accounting Method}
In the accounting method of amortized complexity analysis, we \textit{assign}
a cost to each operation - this amortized cost does not necessarily reflect
the actual cost of an operation. When an operation's amortized cost exceeds
its actual cost, we perceive this as deposited credit that we can use to pay
for subsequent operations that might be undercharged, that is an operation
whose amortized cost is less than its actual cost.

Although this method is very similar to \textit{aggregate analysis}, the key
difference between these is that in aggregate analysis all operations have
the same amortized cost, whereas in the accounting method we may assign
different amortized cost for different operations.

It follows naturally in aggregate analysis that the total amortized cost of
$n$ operations provides an upper bound on the total actual cost of the
sequence, but for the accounting method we must define this explicitly.
\begin{align}
	\sum_{i=1}^{n}{\hat{c_i}} \geq \sum_{i=1}^{n}{c_i}
	\quad \text{, where }c_i\text{ is the actual cost, and }
	\hat{c_i}\text{ is the amortized cost.}
\end{align}
This relationship, as described by the above equation, must hold for all
sequences of operations.

\subsection{Example}
...

\section{Potential Method}
...

\subsection{Example}
...

