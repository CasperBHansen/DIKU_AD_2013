%========== Amortized Complexity ==========%

\chapter{Amortized Complexity}
\label{ch:amortizedcomplexity}

\textbf{Relevant Assignment} Week ?, Problem ?-?\\\\
\textbf{Keywords} Aggregate analysis, accounting- and potential method, average.
\vspace{1in}

\noindent In amortized complexity analysis, we average the time required to
perform a sequence of data-structure operations over all the operations
performed. With amortized analysis, we can show that the average cost of an
operation is small, if we average over a sequence of operations, even though
a single operation within the sequence might be expensive.
\\\\
\noindent \textbf{Amortized- and Average-case Analysis} \\
Amortized analysis differs from average-case analysis in that probability is
not involved; an amortized analysis guarantees the \textit{average performance
of each operation in the worst case}.

\newpage
\section{Aggregate Analysis}
In aggregate analysis, we show that for all $n$, a sequence of $n$ operations
takes \textit{worst-case} time $T(n)$ in total. The average cost, or \textbf{
amortized cost}, however, per operation is $\frac{T(n)}{n}$. The amortized
cost applies to each operation, even when there are several types of
operations in the sequence.

\subsection{Example}
...

\section{Accounting Method}
...

\subsection{Example}
...

\section{Potential Method}
...

\subsection{Example}
...

