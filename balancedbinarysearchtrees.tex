%========== Balanced Binary Search Trees ==========%

\chapter{Balanced Binary Search Trees}
\label{ch:balancedbinarysearchtrees}

\textbf{Relevant Assignment} Problem 13-3\\\\
\textbf{Pensum} CLRS Ch. 12 + 13\\\\
\textbf{Algorithms} Red-black, AVL\\\\
\textbf{Keywords} Rotation
\vspace{1in}

\noindent A binary search tree is a node-based data structure that uses
pointers to keep the structure of a a set of nodes intact. Since the data
structure is purely held together by pointers performing alterations on it
is very fast.

\newpage
\section{Insertion}
...

\section{Deletion}
...

\section{Rotation}
...

\section{Rebalancing}
When we use a balanced implementation of binary search trees, such as
\textit{red-black}- or \textit{AVL}-trees, we must maintain certain
properties, which are typically accounted for after applying an alteration on
the tree - this is called \textit{rebalancing}.

