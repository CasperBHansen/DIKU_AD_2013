%========== Minimum Spanning Trees ==========%

\chapter{Minimum Spanning Trees}
\label{ch:minimumspanningtrees}

\textbf{Relevant Assignment} Problem 16-1\\\\
\textbf{Pensum} CLRS Ch. 23\\\\
\textbf{Algorithms} Kruskal's, Prim's\\\\
\textbf{Keywords} Greedy, representation invariant, binary- and fibonacci-heap
\vspace{1in}

\noindent A minimum spanning tree $T$ is a is a tree or graph whose edges are
a subset $T \subseteq E$ of a graph $G = (V, E)$, with weight function $w$,
such that the weight of an edge $(u, v) \in E$ is determined by $w$, and the
weight of the tree $T$ is minimized.

The term \textit{spanning} refers to the fact that the tree $T$ spans the
entire set of vertices $V$, and the term \textit{minimum} refers to that the
tree $T$ is of minimum weight.

\newpage
\section{Generic Minimum Spanning Tree}
% p. 626, CLRS
...
\begin{algorithm}
	\label{alg:generic-mst}
	\caption{GenericMST}
	
	\SetKwInOut{Input}{Input}
	\SetKwInOut{Output}{Output}
	
	\SetKwFunction{GenericMST}{GenericMST}
	
	\Input{A graph $G = (V, E)$ and a weight function $w$ defined on $E$.}
	\Output{A minimum spanning tree of $G$.}
	
	\BlankLine
	
	\GenericMST($G$, $w$) \\
	\Begin
	{
		$A = \emptyset$ \\
		\While{$A$ \textnormal{does not form a spanning tree}}
		{
			find an edge $(u, v)$ that is safe for $A$ \\
			$A = A \cup \{(u, v)\}$
		}
	}
\end{algorithm}
\subsection{Loop Invariant}
We state the loop invariant of \ref{alg:generic-mst}; prior to each iteration,
$A$ is a subset of some minimum spanning tree.
\\\\
\noindent \textbf{Initialization} \\
After line 3, where we initialize the set $A$ to the empty set, $A$ trivially
satisfies the loop invariant, as the empty set is a subset of any set.
\\\\
\noindent \textbf{Maintenance} \\
Lines 4-7 maintains the loop invariant, because we only add \textit{safe}
edges to $A$. That is, edges $\{u, v\}$ where $A \cup \{(u, v)\}$ also
satisfies the loop invariant.
\\\\
\noindent \textbf{Termination} \\
Since all edges added to $A$ are in a minimum spanning tree, $A$ must be a
minimum spanning tree of $G$.

\newpage
\section{Kruskal's Algorithm}
% p. 631
...

% TODO: pseudo-code

\section{Prim's Algorithm}
% p. 634
...

% TODO: pseudo-code
