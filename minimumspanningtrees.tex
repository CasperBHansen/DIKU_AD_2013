%========== Minimum Spanning Trees ==========%

\chapter{Minimum Spanning Trees}
\label{ch:minimumspanningtrees}

\textbf{Pensum} 23 \cite{clrs} \\\\
\textbf{Assignments} 13-3 \\\\
\textbf{Algorithms} Kruskal's algorithm, Prim's algorithm \\\\
\textbf{Keywords} Greedy, optimal substructure, light edges,
binary- and fibonacci-heap
\vspace{1in}

\noindent A minimum spanning tree $T$ is a is a tree or graph whose edges are
a subset $T \subseteq E$ of a weighted graph $G = (V, E)$, with weight
function $w$, such that the weight of an edge $(u, v) \in E$ is determined by
$w$, and the weight of the tree $T$ is minimized.

The total weight of a graph $T$ is given by the sum of all of its edge weights.
\begin{align}
	w(T) = \sum_{(u, v) \in T}w(u, v)
\end{align}

The term \textit{spanning} refers to the fact that the tree $T$ spans the
entire set of vertices $V$, and the term \textit{minimum} refers to that the
tree $T$ is of minimum weight.
\section{Definitions}
These are the definitions specific to the minimum spanning tree problem.
\begin{description}
	\item A \textbf{cut} $(S, V-S)$ of an undirected graph $G = (V, E)$ is a
partition of $V$.
	\item A \textbf{crossing} is an edge $(u, v) \in E$ that \textit{crosses}
the cut $(S, V - S)$ --- that is, the edges end-points are in $S$ and $S - V$,
respectively.
	\item An cut \textbf{respects} a set $A$ of edges if no edges in $A$
crosses the cut.
	\item A \textbf{light edge} is an edge that crosses a cut and is of
minimum weight (see theorem 23.1\cite[p. 626]{clrs}).
\end{description}

\section{Optimal Substructure}
...

\section{Overlapping Subproblems}
...
% TODO: prove that removing an edge (u, v) yields two separate trees.

\newpage
\section{Generic Minimum Spanning Tree}
% p. 626, CLRS
The algorithm given below for finding the minimum spanning tree is a
generalization, meaning that it is somewhat loosely defined. As such, this
is merely a template for building an actual algorithm.
\begin{algorithm}
	\SetKwInOut{Input}{Input}
	\SetKwInOut{Output}{Output}
	
	\SetKwFunction{GenericMST}{GenericMST}
	
	\Input{A graph $G = (V, E)$ and a weight function $w$ defined on $E$.}
	\Output{A minimum spanning tree of $G$.}
	
	\BlankLine
	
	\GenericMST($G$, $w$) \\
	\Begin
	{
		$A = \emptyset$ \\
		\While{$A$ \textnormal{does not form a spanning tree}}
		{
			find an edge $(u, v)$ that is safe for $A$ \\
			$A = A \cup \{(u, v)\}$
		}
	}
	
	\caption{GenericMST}
	\label{alg:generic-mst}
\end{algorithm}
\subsection{Loop Invariant}
We state the loop invariant of algorithm \ref{alg:generic-mst}; prior to each
iteration, $A$ is a subset of some minimum spanning tree.
\\\\
\noindent \textbf{Initialization} \\
After line 3, where we initialize the set $A$ to the empty set, $A$ trivially
satisfies the loop invariant, as the empty set is a subset of any set.
\\\\
\noindent \textbf{Maintenance} \\
Lines 4-7 maintains the loop invariant, because we only add \textit{safe}
edges to $A$. That is, edges $(u, v)$ where $A \cup \{(u, v)\}$ also
satisfies the loop invariant.
\\\\
\noindent \textbf{Termination} \\
Since all edges added to $A$ are in a minimum spanning tree, $A$ must be a
minimum spanning tree of $G$.

\newpage
\section{Kruskal's Algorithm}
% p. 631
The algorithm follows the greedy strategy by sorting the edges into
nondecreasing order of their weights and then simply adding each edge to
the forest if it does not already belong to it.
\\\\
\begin{algorithm}[H]
	\caption{Kruskal's algorithm}
	\label{alg:kruskal-mst}
	
	\SetKwInOut{Input}{Input}
	\SetKwInOut{Output}{Output}
	
	\SetKwFunction{KruskalMST}{KruskalMST}
	\SetKwFunction{MakeSet}{MakeSet}
	\SetKwFunction{FindSet}{FindSet}
	\SetKwFunction{Union}{Union}
	
	\Input{A graph $G$ and a weight function $w$.}
	\Output{A minimum spanning tree.}
	
	\BlankLine
	\KruskalMST($G$, $w$) \\
	\Begin
	{
		% initialization O(V)
		$A = \emptyset$ \\
		\For{\textnormal{each vertex } $v \in G.V$}
		{
			\MakeSet($v$)
		}
		Sort the edges of $G.E$ into nondecreasing order by weight $w$ \\
		
		% maintenance
		\For{\textnormal{each edge } $(u, v) \in G.E$}
		{
			\If{\FindSet$(u)$ $\neq$ \FindSet$(v)$}
			{
				$A = A \cup \{(u, v)\}$ \\
				\Union($u$, $v$)
			}
		}
		\Return $A$
	}
\end{algorithm}
\begin{description}
	\item \texttt{MakeSet} creates a set whose only member and representative is $x$ --- $\Theta(1)$.
	\item \texttt{FindSet} returns the representative of the set containing $x$ --- $\Theta(1)$.
	\item \texttt{Union} creates the union set --- $\Theta(1)$.
\end{description}

%\newpage
\subsection{Analysis}
%The running-time of \ref{alg:kruskal-mst} depends on the implementation of
%the disjoint-set data structure. Assuming the use of the fastest
%implementation known (disjoint-set-forest, see section
%21.3\cite[p. 568]{clrs}), the running-time is as follows.
%
The initialization occurs on lines 3--7. Line 3 takes constant time
$\Theta(1)$, lines 4--6 performs $|V|$ \texttt{MakeSet} operations. We can
sort the edges $G.E$ in $O(E \lg E)$-time.

The loop on lines 8--13 performs $O(E)$ \texttt{FindSet}- and \texttt{Union}
operations giving us grand a total of $O((V+E) \alpha(V))$, where $\alpha$
is a slowly growing function (see section 21.4\cite[p. 573]{clrs}).

\newpage
\section{Prim's Algorithm}
% p. 634
This algorithm relies on a priority queue (see section
\ref{ch:heaps|sub:priorityqueues} on page \pageref{ch:heaps|sub:priorityqueues}
for more on this topic). It works by building a min-heap priority queue of the
vertices, and then continually extracting the minimum off of the heap. On each
extraction, the algorithm goes through the adjacency list and updates the key
and parent of the vertices still in the priority queue, which sets us up for
the next iteration.
\\\\
\begin{algorithm}[H]
	\caption{Prim's algorithm}
	\label{alg:prim-mst}
	
	\SetKwInOut{Input}{Input}
	\SetKwInOut{Output}{Output}
	
	\SetKwFunction{PrimMST}{PrimMST}
	\SetKwFunction{ExtractMin}{ExtractMin}
	
	\SetKw{Nil}{NIL}
	
	\Input{A graph $G$, a weight function $w$ and an arbitrary root $r$.}
	\Output{In-place minimized spanning tree of $G$.}
	
	\BlankLine
	\PrimMST($G$, $w$, $r$) \\
	\Begin
	{
		% initialization O(V)
		\For{$u \in G.V$}
		{
			$u.key = \infty$ \\
			$u.\pi = $ \Nil
		}
		$r.key = 0$ \\
		$Q = G.V$ \\
		
		% ... O(?)
		\While{$Q \neq \emptyset$}
		{
			$u = \ExtractMin(Q)$ \\
			\For{$v \in G.Adj[u]$}
			{
				\If{$v \in Q$ \textnormal{and} $w(u, v) < v.key$}
				{
					$v.\pi = u$ \\
					$v.key = w(u, v)$
				}
			}
		}
	}
\end{algorithm}

\subsection{Analysis}
Assuming the use of the algorithm described in
\ref{ch:heaps|sub:priorityqueues}, then the initialization part on lines 3--8
contributes $O(V)$ to the running-time.

The loop on lines 9--17 are run exactly $|V|$ times. Since each
\texttt{ExtractMin} costs $O(\lg V)$, this loop contributes $O(V \lg V)$. The
loop on lines 11--16 runs exactly $|E|$ times. We maintain a priority queue
membership bit, such that the check is constant time $\Theta(1)$. The key
update is an implicit \texttt{DecreaseKey}, taking $O(\lg V)$, and thus we
have $O(V \lg V + E \lg V) = O(E \lg V)$.

\subsection{Proof}
By the generic minimum spanning tree algorithm \ref{alg:generic-mst}, Prim's
algorithm build on its principles, and as such implicitly maintains the set
$A$ from the generic algorithm as
\begin{align}
	A = \{ (v, v.\pi) : v \in V - \{r\} - Q \}
\end{align}
When the algorithm terminates, the priority queue has been exhausted and the
minimum spanning tree $A$ for $G$ is thus
\begin{align}
	A = \{ (v, v.\pi) : v \in V - \{r\} \}
\end{align}
The algorithm maintains a three-part loop invariant:
\begin{enumerate}
	\item $A = \{ (v, v.\pi) : v \in V - \{r\} - Q \}$.
	\item The vertices already placed into the MST are those in $V - Q$.
	\item For all vertices $v \in Q$, if $v.\pi \neq $ \texttt{nil}, then
$v.key < \infty$ and $v.key$ is the weight of a light edge $(v, v.\pi)$
connecting $v$ to some vertex already in the MST.
\end{enumerate}
Line 10 identifies a vertex $u \in Q$ incident on a light edge that crosses
the cut $(V - Q, Q)$ --- except the first, due to line 7. Removing $u$ from
$Q$ adds it to the set $V - Q$, thus $(u, u.\pi)$ becomes a member of $A$.
The inner loop maintains the third invariant by updating the $key$ and $\pi$
attributes of every vertex adjacent to $u$ that is not in the tree.
