%========== Minimum Spanning Trees ==========%

\chapter{Minimum Spanning Trees}
\label{ch:minimumspanningtrees}

\textbf{Relevant Assignment} Problem 16-1\\\\
\textbf{Pensum} CLRS Ch. 23\\\\
\textbf{Algorithms} Kruskal's, Prim's\\\\
\textbf{Keywords} Greedy, binary-heap, fibonacci-heap
\vspace{1in}

\noindent A minimum spanning tree $T$ is a is a tree or graph whose edges are
a subset $T \subseteq E$ of a graph $G = (V, E)$, with weight function $w$,
such that the weight of an edge $(u, v) \in E$ is determined by $w$, and the
weight of the tree $T$ is minimized.

The term \textit{spanning} refers to the fact that the tree $T$ spans the
entire set of vertices $V$, and the term \textit{minimum} refers to that the
tree $T$ is of minimum weight.

\newpage
\section{Generic Minimum Spanning Tree}
% p. 626, CLRS
...
% TODO: pseudo-code of Generic-MST

% TODO: state the loop-invariant

% TODO: prove the loop-invariant holds

\section{Kruskal's Algorithm}
% p. 631
...

% TODO: pseudo-code

\section{Prim's Algorithm}
% p. 634
...

% TODO: pseudo-code
